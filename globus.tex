\documentclass[slidestop,mathserif,compress,xcolor=svgnames]{beamer} 
\mode<presentation>
{  
%   \setbeamertemplate{background canvas}[vertical shading][bottom=DarkBlue!5,top=DarkBlue!5]
  % \setbeamertemplate{background canvas}[vertical shading][bottom=red!1,top=red!1]
%  \setbeamertemplate{navigation symbols}[vertical]
  \setbeamertemplate{navigation symbols}{}%{\insertsectionnavigationsymbol}
  \usetheme{Pittsburgh}
   \usecolortheme{tigers}
   \useinnertheme{rounded}
% \usecolortheme{bayou}
  \usefonttheme[onlysmall]{structureitalicserif}
%  \setbeamertemplate{blocks}[rounded][shadow=true]
  
\usefoottemplate{\vbox{%
\tinycolouredline{structure}%
{\color{tigersgold}\textbf{\insertshortauthor\hfill\insertshortdate}}%
}}
}

\usepackage{pgf,pgfarrows,pgfnodes,pgfautomata,pgfheaps,pgfshade}
\usepackage{amsmath,amssymb,subfigure}
\usepackage[latin1]{inputenc}
\usepackage{colortbl}
\usepackage[english]{babel}
\usepackage{hyperref}
\usepackage{movie15}
\hypersetup{
  pdftitle={Globus Tutorial},
  pdfauthor={Alexander B. Pacheco, User Services Consultant, Louisiana State University}
}
%\usepackage{movie15}
\usepackage{times}

\setbeamercovered{dynamic}
\beamersetaveragebackground{DarkBlue!2}
\beamertemplateballitem

\usepackage[english]{babel}
\usepackage[latin1]{inputenc}
\usepackage{times}
\usepackage{amsmath}
\usepackage[T1]{fontenc}
\usepackage{graphicx}
\title[Globus Procedures ]{Globus Procedures on LONI Systems}


\author[Alex Pacheco]{Alexander~B.~Pacheco}
       
\institute[HPC@LSU \& LONI] {\inst{}User Services Consultant\\LSU HPC \& LONI\\sys-help@loni.org}

\date[LONI HPC Workshop - ULL, Nov 1, 2010]{\tiny{LONI HPC Workshop\\University of Louisiana at Lafayette\\November 1, 2010}}
     
\subject{Talks}
% This is only inserted into the PDF information catalog. Can be left
% out. 




% If you have a file called "university-logo-filename.xxx", where xxx
% is a graphic format that can be processed by latex or pdflatex,
% resp., then you can add a logo as follows:

\pgfdeclareimage[height=1.5cm]{loni-logo}{LONI}
\pgfdeclareimage[height=0.5cm]{gold-logo}{GOLD_LSU}
\pgfdeclareimage[height=0.5cm]{purple-logo}{PURPLELSU}
\pgfdeclareimage[height=1.5cm]{lsuccttower-logo}{FULLCOLORTOWER_VERT}%{qtp}
\pgfdeclareimage[height=0.5cm]{lsutower-logo}{ProcessHorizontal}
\logo{\pgfuseimage{lsuccttower-logo}}

% put MSRI logo in bottom left
\setbeamertemplate{sidebar left}{%
   \vfill%
   \rlap{\hskip0.1cm%
         %
         {\pgfuseimage{loni-logo}}}%
   \vskip2pt%
   \llap{\usebeamertemplate***{navigation symbols}\hskip0.1cm}%
   \vskip2pt%
}

% Delete this, if you do not want the table of contents to pop up at
% the beginning of each subsection:
% \AtBeginSection[]
% {
%   \begin{frame}<beamer>
%     \frametitle{\small{Outline}}
%     \small
%     \tableofcontents[currentsection,currentsubsection]
%   \end{frame}
% }

\begin{document}

\frame{\titlepage}

\normalsize
%\begin{frame}[label=toc,squeeze]
%  \footnotesize
%  \frametitle{\small{Outline}}
%  \tableofcontents
%\end{frame}

\normalsize
% \begin{frame}
%   \frametitle{\small Outline}
%  \begin{itemize}
%   \item Using Globus Procedures on LONI Systems.
%   \begin{enumerate}
%    \item Setting up Globus Environment on LONI.
%    \item Creating Grid Certificate.
%    \item Submitting Jobs using Globus.
%    \item Moving Files using Globus
%   \end{enumerate}
%  \end{itemize}
% \end{frame}


% \part{Introduction}
\section{Introduction}
\begin{frame}[fragile]
\frametitle{\small So Far}
\begin{itemize}
\item Overview of HPC
\item Introduction to vi
\item LONI Programming Environment
\item Ready to use LONI Systems
\end{itemize}
\end{frame}

\begin{frame}[fragile,allowframebreaks]
% \frametitle{{\small So Far}}
\begin{itemize}
\item BUT WHICH ONE

\begin{columns}
\column{4cm}
\begin{block}{LINUX}
\begin{enumerate}
\item Eric
\item Louie
\item Oliver
\item Painter
\item Poseidon
\item Queenbee
\end{enumerate}
\end{block}
\column{4cm}
\begin{block}{AIX}
\begin{enumerate}
\item Bluedawg
\item Ducky
\item Lacumba
\item Neptune
\item Zeke
\end{enumerate}
\end{block}
\end{columns}
\item How do I manage data and jobs on the LONI systems?
\item Do I need to login to all systems or
\item Is there a tool to manage this
\end{itemize}
\end{frame}

\begin{frame}[fragile]
\frametitle{{\small Globus Toolkit}}
\begin{block}{}
\begin{itemize}
\item Globus Toolkit for building computing grids.
\item Provides 
\begin{enumerate}
\item Grid Resource Allocation Manager (GRAM) for managing jobs
\item GridFTP, a more reliable and high performance file transfer for Grid computing applications
\end{enumerate}
\end{itemize}
\end{block}

\begin{exampleblock}{}
\footnotesize{\bf
\url{http://www.cct.lsu.edu/~apacheco/tutorials/globus.php}\\
\url{http://www.cct.lsu.edu/~apacheco/tutorials/globus.pdf}}
\end{exampleblock}
\end{frame}

% \part{Requirements}
\section{Setting up Globus Environment on LONI}
\begin{frame}[fragile]
\frametitle{{\small Requirements}}
\begin{itemize}
\item Access to LONI Account
\begin{itemize}
  \item Apply for LONI accounts at \color{tigerspurple}{\bf\url{https://allocations.loni.org}}
\end{itemize}
\item Active LONI Grid Certificate
\item Login to qb1.loni.org
\tiny{
\color{DarkBlue}{\begin{verbatim}
apacheco@pacheco:~/Tutorials/Globus$ ssh qb1.loni.org
 \end{verbatim}}
\color{DarkGreen}{\vspace{-0.5cm}\begin{verbatim}
Password: 
Last login: Mon Nov  1 09:49:17 2010 from 130.70.53.27
 \end{verbatim}}
}
\item Add \texttt{globus} to your environment and resoft
{\tiny
\color{DarkBlue}{\begin{verbatim}
[apacheco@qb1 ~]$ soft add +globus-4.0.8-r2
[apacheco@qb1 ~]$ resoft
\end{verbatim}}
}
\end{itemize}
\end{frame}

\normalsize
\section{Creating Grid Certificate}
\begin{frame}[fragile]
\frametitle{\small{How To Request for a Grid Certificate}}
\begin{itemize}
 \item Request LONI Grid Certificate
{\tiny
\color{DarkBlue}{\begin{verbatim}
[apacheco@qb1 ~]$ grid-cert-request
 \end{verbatim}}
\color{DarkGreen}{\vspace{-0.5cm}\begin{verbatim}
A certificate request and private key is being created.
You will be asked to enter a PEM pass phrase.
This pass phrase is akin to your account password,
and is used to protect your key file.
If you forget your pass phrase, you will need to
obtain a new certificate.

Generating a 1024 bit RSA private key
..................................++++++
............++++++
writing new private key to '/home/apacheco/.globus/userkey.pem'
Enter PEM pass phrase:
 \end{verbatim}}
}
\item To recreate a certificate
{\tiny
\color{DarkBlue}{\begin{verbatim}
[apacheco@qb1 ~]$ grid-cert-request -force
\end{verbatim}}
}
\end{itemize}
\end{frame}
\begin{frame}[fragile,allowframebreaks]
\frametitle{\small{Grid Certificates}}
\begin{itemize}
\item .globus directory will be created in the home directory with three files:
\texttt{usercert$\_$request.pem},\texttt{userkey.pem} and \texttt{usercert.pm}
\item email \texttt{usercert$\_$request.pem} file to {\color{tigerspurple}{\footnotesize\bf{\href{mailto:ca@loni.org}{ca@loni.org}}}}
{\tiny
\color{DarkBlue}{\begin{verbatim}
[apacheco@qb1 ~]$ cat $HOME/.globus/usercert_request.pem | mail ca@loni.org
\end{verbatim}}
}
\item Copy the signed certificate you receive from LONI administrator to
\texttt{$\sim$/.globus/usercert.pem}
\vspace{5cm}
\item {Certificate Information}
{\tiny
\color{DarkBlue}{\begin{verbatim}
[apacheco@qb1 ~]$ grid-cert-info
 \end{verbatim}}
\color{DarkGreen}{\vspace{-0.5cm}\begin{verbatim}
Certificate:
    Data:
        Version: 3 (0x2)
        Serial Number: 340 (0x154)
        Signature Algorithm: md5WithRSAEncryption
        Issuer: C=US, O=Louisiana Optical Network Initiative, OU=loni.org, CN=LONI CA
 blah blah blah
\end{verbatim}}
}
\item {Distinguished Name}
{\tiny
\color{DarkBlue}{\begin{verbatim}
[apacheco@qb1 ~]$ grid-cert-info -subject
 \end{verbatim}}
\color{DarkGreen}{\vspace{-0.5cm}\begin{verbatim}
/C=US/O=Louisiana Optical Network Initiative/OU=loni.org/OU=sys.loni.org/CN=Alexander Pacheco
\end{verbatim}}
}
\item {Issuer Hash}
{\tiny
\color{DarkBlue}{\begin{verbatim}
[apacheco@qb1 ~]$ grid-cert-info -issuerhash
 \end{verbatim}}
\color{DarkGreen}{\vspace{-0.5cm}\begin{verbatim}
a3bf9f3c
\end{verbatim}}
}
\vspace{5cm}
\item {More Details/Help}
{\tiny
\color{DarkBlue}{\begin{verbatim}
[apacheco@qb1 ~]$ grid-cert-info -help
 \end{verbatim}}
\color{DarkGreen}{\vspace{-0.5cm}\begin{verbatim}
grid-cert-info [-help] [-file certfile] [-all] [-subject] [...]

    Displays certificate information. Unless the optional -file
    argument is given, the default location of the file containing the
blah blah blah
\end{verbatim}
}}
\item Generating a valid proxy.
\end{itemize}
{\tiny
\color{DarkBlue}{\begin{verbatim}
[apacheco@qb1 ~]$ grid-proxy-init
 \end{verbatim}}
\color{DarkGreen}{\vspace{-1.0cm}\begin{verbatim}
Your identity: /C=US/O=Louisiana Optical Network Initiative/OU=loni.org/OU=sys.loni.org/CN=Alexander Pacheco
Enter GRID pass phrase for this identity:
Creating proxy ............................................................... Done
Your proxy is valid until: Fri Oct 22 03:30:20 2010
\end{verbatim}}
}
\end{frame}

% \part{File Transfer}
\section{File Transfer using Globus}
\begin{frame}[fragile,allowframebreaks]
 \frametitle{\small Moving Files using Globus}
\begin{itemize}
 \item Globus provides \texttt{globus-url-copy}, a scriptable command line tool that can do multi-protocol data movement.
 \item Supports: gsiftp:// (GridFTP), ftp://, http://, https://, and file:///
 \item Usage: 
{\tiny
\color{DarkBlue}{\begin{verbatim}
globus-url-copy [-help | -usage] [-version[s]] [-vb] [-dbg] [-b | -a]
                [-q] [-r] [-rst] [-f <filename>]
                [-s <subject>] [-ds <subject>] [-ss <subject>]
                [-tcp-bs <size>] [-bs <size>] [-p <parallelism>]
                [-notpt] [-nodcau] [-dcsafe | -dcpriv]
                <sourceURL> <destURL>
\end{verbatim}}
}
\item transfer a file from queenbee to oliver
{\tiny
\color{DarkBlue}{\begin{verbatim}
[apacheco@qb1 globus-tutorial]$ globus-url-copy \
     gsiftp://qb1.loni.org/home/apacheco/globus-tutorial/100mbfile \
     gsiftp://oliver1.loni.org/home/apacheco/100mbfile
\end{verbatim}}
}
\item create a 1GB file and transfer to zeke with verbose (-vb) option
{\tiny
\color{DarkBlue}{\begin{verbatim}
[apacheco@qb1 globus-tutorial]$ dd count=1000 bs=1024k if=/dev/zero \
 of=/work/apacheco/1gbfile
 \end{verbatim}}
\color{DarkGreen}{\vspace{-0.5cm}\begin{verbatim}
1000+0 records in
1000+0 records out
\end{verbatim}}
\color{DarkBlue}{\begin{verbatim}
[apacheco@qb1 globus-tutorial]$ globus-url-copy -vb \
       file:///work/apacheco/1gbfile \
       gsiftp://zeke.loni.org/mnt/lpfs.nfs302/apacheco/1gbfile
 \end{verbatim}}
\color{DarkGreen}{\vspace{-0.5cm}\begin{verbatim}
Source: file:///work/apacheco/
Dest:   gsiftp://zeke.loni.org/mnt/lpfs.nfs302/apacheco/
  1gbfile
    1020264448 bytes        29.87 MB/sec avg        35.00 MB/sec inst
\end{verbatim}
}}
\item Create a directory to store a filename
{\tiny
\color{DarkBlue}{\begin{verbatim}
[apacheco@qb1 ~]$ globus-url-copy -vb -cd \
gsiftp://eric1.loni.org/home/apacheco/100mbfile \
gsiftp://qb1.loni.org/home/apacheco/createdirectory/100mbfile
 \end{verbatim}}
\color{DarkGreen}{\vspace{-0.5cm}\begin{verbatim}
Source: gsiftp://eric1.loni.org/home/apacheco/
Dest:   gsiftp://qb1.loni.org/home/apacheco/createdirectory/
  100mbfile
    104857600 bytes        58.82 MB/sec avg        58.82 MB/sec inst 
\end{verbatim}
}}
\item Transfer files from a directory recursively
{\tiny
\color{DarkBlue}{\begin{verbatim}
 [apacheco@qb1 globus-tutorial]$ globus-url-copy -vb -r -cd \
gsiftp://qb1.loni.org/home/apacheco/globus-tutorial/ \
gsiftp://oliver1.loni.org/home/apacheco/createdirectory/
 \end{verbatim}}
\color{DarkGreen}{\vspace{-0.5cm}\begin{verbatim}
Source: gsiftp://qb1.loni.org/home/apacheco/globus-tutorial/
Dest:   gsiftp://oliver1.loni.org/home/apacheco/createdirectory/
  200mbfile
    209715200 bytes        34.48 MB/sec avg        34.48 MB/sec inst
  100mbfile
    104857600 bytes        27.03 MB/sec avg        27.03 MB/sec inst
\end{verbatim}
}}
\item You can adjust the tcp buffer size and buffer size using the options \texttt{-tcp-bs} and \texttt{-bs}
\item Use parallel streams for transfer with \texttt{-p} option
{\tiny
\color{DarkBlue}{\begin{verbatim}
[apacheco@qb1 globus-tutorial]$ globus-url-copy -p 4 -vb \
     file:///work/apacheco/1gbfile \
     gsiftp://zeke.loni.org/mnt/lpfs.nfs302/apacheco/1gbfile
\end{verbatim}}
}
\item On the LONI machines, \texttt{-p 5 -tcp-bs 2097152} is recommended for a reasonable performance (not necessarily the most optimized set of option numbers).
\item multiple transfers from a script file using the \texttt{-f} switch (example in job submission section)
\end{itemize}

\end{frame}
% \part{Submit Jobs}
\section{Submitting Jobs using Globus}
\subsection{Serial}
\begin{frame}[fragile,allowframebreaks]
\frametitle{\small{Job Submission}}
\begin{itemize}
\item Job Submission using \texttt{globus-job-run}.
{\tiny
\color{DarkBlue}{\begin{verbatim}
[apacheco@qb1 ~]$ globus-job-run -help
 \end{verbatim}}
\color{DarkGreen}{\vspace{-0.5cm}\begin{verbatim}
Usage: globus-job-run
      [-help|-usage]                     print usage and exit
      [-version]                         print version and exit

\end{verbatim}
}}
\item Run a simple command on a LONI machine 
{\tiny
\color{DarkBlue}{\begin{verbatim}
[apacheco@qb1 ~]$ globus-job-run louie1.loni.org /bin/date
 \end{verbatim}}
\color{DarkGreen}{\vspace{-0.5cm}\begin{verbatim}
Thu Oct 21 14:57:19 CDT 2010
\end{verbatim}
}
\color{DarkBlue}{\begin{verbatim}
[apacheco@qb1 ~]$ globus-job-run louie1.loni.org /bin/hostname
 \end{verbatim}}
\color{DarkGreen}{\vspace{-0.5cm}\begin{verbatim}
louie1.loni.org
\end{verbatim}
}}
\item Run a shell script with arguments using globus
{\tiny
\color{DarkBlue}{\begin{verbatim}
[apacheco@qb1 ~]$ globus-job-run oliver1.loni.org -s random-generator.sh 4 7
 \end{verbatim}}
\color{DarkGreen}{\vspace{-0.5cm}\begin{verbatim}
2377875
2910657
8458138
1522082
 \end{verbatim}}
\vspace{-0.5cm}
\color{DarkBlue}{\begin{verbatim}
[apacheco@qb1 ~]$ globus-job-run bluedawg.loni.org -s random-generator.sh 4 7
 \end{verbatim}}
\color{DarkGreen}{\vspace{-0.5cm}\begin{verbatim}
5138706
1757255
3086336
5345316
\end{verbatim}
}}
\vspace{5cm}
\item -np N : number of processing elements
\item -stdin [-l|-s] file  : standard input 
\item -stdout [-l|-s] file : standard output 
\item -l[ocal] : file is relative to working directory of job (DEFAULT)
\item -s[tage] : file relative to job request is staged to job host
\item -x rsl-clause : RSL extension capability
\vspace{4cm}
\item Execute commands on remote machine using input from staging machine
{\tiny
\color{DarkBlue}{\begin{verbatim}
[apacheco@qb1 ~]$ globus-job-run louie1.loni.org -stdin -s /etc/hosts -l /bin/cat 
 \end{verbatim}}
\color{DarkGreen}{\vspace{-0.5cm}\begin{verbatim}
# Do not remove the following line, or various programs
# that require network functionality will fail.
127.0.0.1       localhost.localdomain localhost
204.90.40.60    eta.hpc.lsu.edu eta
208.100.92.71   l6l1u01.sys.loni.org
.....
10.192.92.247    qbib20-1
10.192.92.248    qbib20-2 
\end{verbatim}
}}
\item Execute commands on remote machine using input from local machine
{\tiny
\color{DarkBlue}{\begin{verbatim}
[apacheco@qb1 ~]$ globus-job-run louie1.loni.org -stdin -l /etc/hosts -l /bin/cat
 \end{verbatim}}
\color{DarkGreen}{\vspace{-0.5cm}\begin{verbatim}
# Do not remove the following line, or various programs
# that require network functionality will fail.
127.0.0.1       localhost.localdomain localhost
204.90.40.60    eta.hpc.lsu.edu eta
208.100.92.71   l6l1u01.sys.loni.org
208.100.92.72   l6l1u02.sys.loni.org
.....
172.17.73.227  louie127h
172.17.73.228  louie128h
\end{verbatim}
}}
\vspace{5cm}
\item Execute commands and write output on staging machine
{\tiny
\color{DarkBlue}{\begin{verbatim}
[apacheco@qb1 ~]$ globus-job-run louie1.loni.org -stdout -s hellohosts \
-stdin -s /etc/hosts -l /bin/cat
[apacheco@qb1 ~]$ head -4 hellohosts ; tail -2 hellohosts 
 \end{verbatim}}
\color{DarkGreen}{\vspace{-0.5cm}\begin{verbatim}
# Do not remove the following line, or various programs
# that require network functionality will fail.
127.0.0.1       localhost.localdomain localhost
204.90.40.60    eta.hpc.lsu.edu eta
10.192.92.247    qbib20-1
10.192.92.248    qbib20-2
\end{verbatim}
}}
 \item Job submission using \texttt{globus-job-submit}
 \item submit a job to the scheduling jobmanager
 {\tiny
   \color{DarkBlue}{\begin{verbatim}
 [apacheco@qb1 ~]$ globus-job-submit oliver1.loni.org/jobmanager /bin/date
\end{verbatim}}
  \color{DarkGreen}{\vspace{-0.0cm}\begin{verbatim}
https://oliver1.loni.org:50886/26507/1288375416/
\end{verbatim}}
}
 \item \texttt{globus-job-status} checks status of jobmanager
 {\tiny
   \color{DarkBlue}{\begin{verbatim}
[apacheco@qb1 ~]$ globus-job-status https://oliver1.loni.org:50886/26507/1288375416/
\end{verbatim}}
  \color{DarkGreen}{\vspace{-0.0cm}\begin{verbatim}
DONE
\end{verbatim}}
}
 \item \texttt{globus-job-get-output} gets output of the job
 {\tiny
   \color{DarkBlue}{\begin{verbatim}
[apacheco@qb1 ~]$ globus-job-get-output https://oliver1.loni.org:50886/26507/1288375416/
\end{verbatim}}
  \color{DarkGreen}{\vspace{-0.0cm}\begin{verbatim}
Fri Oct 29 13:03:36 CDT 2010
\end{verbatim}}
}
\end{itemize}
\end{frame}

\subsection{Parallel}
\begin{frame}[fragile,allowframebreaks]
 \frametitle{\small{Submit Parallel Jobs}}
\begin{itemize}
 \item Submit an openmp parallelized code on
\begin{enumerate}
\item eric (PBS)
{\tiny
\color{DarkBlue}{\begin{verbatim}
[apacheco@qb1 ~]$ globus-job-run eric1.loni.org/jobmanager-pbs \
-np 1 -m 10 -p loni_loniadmin1 -q single -stdin \
-s /home/apacheco/dft/flux-full.dat -stdout \
-s flux-eric1-parallel.dat -s /home/apacheco/dft/dft.parallel
\end{verbatim}}
}
\item  zeke (loadleveler)
{\tiny
\color{DarkBlue}{\begin{verbatim}
[apacheco@qb1 ~]$ globus-job-run zeke.loni.org/jobmanager-loadleveler \
 -env "GBLL_NETWORK_MPI=sn_all,not_shared,US" -stdin \
-s /home/apacheco/dft/flux-full.dat -x  "&(jobType=mpi)(count=2)\
(maxWallTime=10)(queue=checkpt)(stdout=/mnt/lpfs.nfs302/apacheco/dft.out)\
(stderr=/mnt/lpfs.nfs302/apacheco/dft.err)(hostCount=1)" \
-s /home/apacheco/dft/dft.xlf
[apacheco@qb1 ~]$ cat dft.output 
 \end{verbatim}}
\color{DarkGreen}{\vspace{-0.5cm}\begin{verbatim}
gsiftp://zeke.loni.org/mnt/lpfs.nfs302/apacheco/dft.out \
file:///home/apacheco/dft.out
gsiftp://zeke.loni.org/mnt/lpfs.nfs302/apacheco/dft.err \
file:///home/apacheco/dft.err
 \end{verbatim}}
\color{DarkBlue}{\vspace{-0.5cm}\begin{verbatim}
[apacheco@qb1 ~]$ globus-url-copy -vb -f dft.output 
 \end{verbatim}}
 \color{DarkGreen}{\vspace{-0.5cm}\begin{verbatim}
Source: gsiftp://zeke.loni.org/mnt/lpfs.nfs302/apacheco/
Dest:   file:///home/apacheco/
  dft.out

Source: gsiftp://zeke.loni.org/mnt/lpfs.nfs302/apacheco/
Dest:   file:///home/apacheco/
  dft.err
\end{verbatim}}
}
\end{enumerate}
\vspace{10cm}
\item copy \texttt{helloworld.c} from \texttt{queenbee.loni.org:/home/apacheco/helloworld.c}
\item compile on bluedawg/zeke/(any other P5 machine) using
{\tiny
\color{DarkBlue}{\begin{verbatim}
mpcc helloworld.c -o helloworld
\end{verbatim}}
}
\item Submit an MPI job 
\begin{enumerate}
\item on bluedawg/zeke/(any other P5 machine) 
{\tiny
\color{DarkBlue}{\begin{verbatim}
[apacheco@qb1 ~]$ globus-job-run bluedawg.loni.org/jobmanager-loadleveler \
-env "GBLL_NETWORK_MPI=sn_all,not_shared,US" -x  "&(jobType=mpi)(count=8)\
(maxWallTime=10)(queue=checkpt)(hostCount=1)" -l /home/apacheco/helloworld
 \end{verbatim}}
\color{DarkGreen}{\vspace{-0.5cm}\begin{verbatim}
Hello world from 0 out of 8
Hello world from 1 out of 8
Hello world from 2 out of 8
Hello world from 3 out of 8
Hello world from 4 out of 8
Hello world from 5 out of 8
Hello world from 6 out of 8
Hello world from 7 out of 8
ATTENTION: 0031-408  8 tasks allocated by LoadLeveler, continuing... 
\end{verbatim}}
}
\item on eric/oliver/(any other intel machine)
{\tiny
\color{DarkBlue}{\begin{verbatim}
[apacheco@qb1 ~]$ globus-job-run eric1.loni.org/jobmanager-pbs \
-x "(jobType=mpi)(hostCount=1)(maxWallTime=10)(queue=checkpt)\
(stdout=/home/apacheco/gljobrun.out)" -s /home/apacheco/helloworld
\end{verbatim}}
\color{DarkGreen}{\vspace{-0.5cm}\begin{verbatim}
[apacheco@eric2 ~]$ cat gljobrun.out
--------------------------------------
Running PBS prologue script
--------------------------------------
User and Job Data:
--------------------------------------
Job ID:    262113.eric2
Username:  apacheco
Group:     loniadmin
Date:      27-Oct-2010 15:44
Node:      eric052 (7032)
--------------------------------------
PBS has allocated the following nodes:

eric052

A total of 4 processors on 1 nodes allocated
---------------------------------------------
Check nodes and clean them of stray processes
---------------------------------------------
Checking node eric052 15:44:34 
Done clearing all the allocated nodes
------------------------------------------------------
Concluding PBS prologue script - 27-Oct-2010 15:44:34
------------------------------------------------------
Hello world from 0 out of 4
Hello world from 2 out of 4
Hello world from 1 out of 4
Hello world from 3 out of 4
\end{verbatim}}
}
\end{enumerate}
\item Job submission using \texttt{globusrun} with Resource Specification Language (RSL) commands in a file

{\tiny
\color{DarkBlue}{\begin{verbatim}
[apacheco@qb1 ~]$ cat eric.rsl                 
\end{verbatim}}
\color{DarkGreen}{\vspace{-0.3cm}\begin{verbatim}
& (jobType=mpi)
(hostCount=1)
(maxWallTime=10)
(queue=checkpt)
(executable=/home/apacheco/helloworld)
(stdout=/home/apacheco/gljobrun.out)                          
\end{verbatim}
}
\color{DarkBlue}{\begin{verbatim}
[apacheco@qb1 ~]$  globusrun -r eric1.loni.org/jobmanager-pbs -f eric.rsl -b            
\end{verbatim}}
\color{DarkGreen}{\vspace{-0.3cm}\begin{verbatim}
globus_gram_client_callback_allow successful
GRAM Job submission successful
https://eric1.loni.org:50886/779/1288371858/
GLOBUS_GRAM_PROTOCOL_JOB_STATE_PENDING                        
\end{verbatim}
}
}
\end{itemize}
\end{frame}


% \begin{frame}[fragile]
%  \frametitle{RSL}
% \begin{itemize}
% \item Resource Specification Language:\\
%   -x ``resource descriptions`` 
% \item Resubmit previous job on eric using \texttt{eric.rsl} and \texttt{globusrun}
% {\tiny
% }
% \end{itemize}
% \end{frame}


\section{References}
\begin{frame}
 \frametitle{{\tiny References}}
\begin{block}{}
\begin{itemize}
 \item \footnotesize{{\bf Globus Toolkit}\\\color{tigerspurple}{\bf \url{http://www.globus.org/toolkit/}}}
 \item \footnotesize{{\bf Loni Docs: Globus Tutorial using LONI resources}\\\color{tigerspurple}{\bf \url{https://docs.loni.org/wiki/Globus_Tutorial_using_LONI_resources}}}
 \item \footnotesize{{\bf Moodle: HPC105 - Getting Started with Globus Toolkit}\\\color{tigerspurple}{\bf \url{https://docs.loni.org/moodle/}}}
\end{itemize}
\end{block}

\end{frame}



\end{document}
